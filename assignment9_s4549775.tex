\documentclass[12pt, a4paper]{article}

\usepackage[backend=biber]{biblatex}
\usepackage{multicol}
\addbibresource{references.bib}

\setlength\parskip{1em}
\setlength\parindent{0em}

\title{Assignment 9}

\author{Hendrik Werner s4549775}

\begin{document}
\maketitle

\clearpage
\section{Feedback}

\begin{tabular}{ll}
	\textbf{Feedback for} & Mick Koomen\\
	\textbf{Exercise} & 8\\
	\textbf{Feedback from} & Hendrik Werner\\
\end{tabular}

\subsection{Abstract}
The abstract mentions the main achievement of the research, passing the Turin Test. It also nicely sums up the contents of the article.

\subsection{Introduction}
You almost literally copied the introduction from exercise 6 and did not incorporate the feedback from exercise 7 yet. The same feedback I gave still applies. I have not processed your feedback either, we should both do so in exercise 10.

Additionally, the beginnings of your abstract and introduction contain the same information just worded differently. I had problems with this too. After reading the abstract, reading the first sentences of your introduction just feels like reading it again.

\begin{multicols}{2}
	\paragraph{Abstract} In this article a new artificial intelligence (AI) is introduced, that is called Wiley. Wiley has passed the Turing Test with a majority vote of 85\%.

	\paragraph{Introduction} In this research paper, a new artificial intelligence (AI), called Wiley, will be
introduced. Wiley has passed the standard Turing test, convincing 85\% of the
judges that partook in the contest.
\end{multicols}

\subsection{Content}
The premise of your article is very interesting, and like you said yourself, people will want to read it because you passed the Turing Test \cite{mick}. You do not explain the least bit how you did this however.

The article starts by covering what a Turin Test is, then there is a section about the general history of AI, and after that you go on to explain the practical applications of Wiley. This is interesting for sure, but I feel like reading this article would leave a large portion of your intended audience unsatisfied, since you say they are "people interested in IT" \cite{mick}.

Had you intended your article for health care workers or psychologists this would have been appropriate. "The article is meant to inform people about the new artificial intelligence (A.I.)
that has passed the Turing test, and how this has been accomplished and with
what technology." \cite{mick} in your own words. Apart from an (empty) section called "Tools", which I assume was meant to be about the technologies you used to develop your AI, you do not write about this at all.

Writing about this is really difficult because you did not actually make an AI that does what you claim in your article \cite{mick}. I do not know how to solve this problem. Nonetheless just going by the titles of your sections it seems like you did not even intend to write about it. I think this is a strange choice given what you said the premise of your article was in exercise 6, and the reasoning you gave about why people would read it.

The sections you actually wrote are good though.

\printbibliography

\end{document}
